\documentclass[oneside,final,14pt]{extarticle}
\usepackage[utf8x]{inputenc}
\usepackage[russianb]{babel}
\usepackage{vmargin}
\setpapersize{A4}
\setmarginsrb{2cm}{1.5cm}{1cm}{1.5cm}{0pt}{0mm}{0pt}{13mm}
\usepackage{indentfirst}
\usepackage{amsmath}
\usepackage{amsfonts}
\usepackage[dvips]{graphicx}
\usepackage{xcolor}
\graphicspath{{pictures/}}
\sloppy

\begin{document}
\begin{titlepage}
\centerline{Национальный Исследовательский Ядерный Университет "<МИФИ">}
\vfill
\Large
\begin{center}
Курсовая работа\\ по общей физике (молекулярная физика\\и статистическая термодинамика) \\ Костенко Ю. А., Зеленев В. С.
\end{center}
\normalsize
\vfill
\begin{flushright}
Выполнили: Костенко Ю. А., \\ Зеленев В. С.
\end{flushright}
\vfill \vfill \vfill
\centerline{Москва - 2024}
\end{titlepage}
\setcounter{page}{2}
\tableofcontents
\newpage

\section{Постановка задач}
\subsection{Задача 1}
Исследовать электродинамику с монополями. Рассмотреть движение диона в однородном электрическом поле $E$; в однородном магнитном поле $B$; в скрещивающихся однородных электрическом и магнитном полях $E$ и $B$, причем считать, что $E \perp B$.
\subsection{Задача 2}
Исследовать модель Изинга для ферромагнетиков. Рассчитать вектор намагниченности, получить петлю гистерезиса (если возможно). 
\subsection{Задача 3}
Изучить движение заряженой частицы в равновесной электронейтральной плазме. Все необходимые параметры плазмы и частицы даны. 
\newpage

\section{Задача 1}

\newpage

\section{Задача 2}

\newpage

\section{Задача 3}


\end{document}