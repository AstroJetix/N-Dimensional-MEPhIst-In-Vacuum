\documentclass[oneside,final,14pt]{extarticle}
\usepackage[utf8x]{inputenc}
\usepackage[russianb]{babel}
\usepackage{vmargin}
\setpapersize{A4}
\setmarginsrb{2cm}{1.5cm}{1cm}{1.5cm}{0pt}{0mm}{0pt}{13mm}
\usepackage{indentfirst}
\usepackage{amsmath}
\usepackage{amsfonts}
\usepackage[dvips]{graphicx}
\usepackage{xcolor}
\graphicspath{{pictures/}}
\sloppy

\begin{document}
	\begin{titlepage}
		\centerline{Национальный Исследовательский Ядерный Университет "<МИФИ">}
		\vfill
		\Large
		\begin{center}
			Курсовая работа\\ по Общей физике (Электричество и магнетизм)
		\end{center}
		\normalsize
		\vfill
		\begin{flushright}
			Выполнили: Костенко Ю. А., \\ Зеленев В. С.
		\end{flushright}
		\vfill \vfill \vfill
		\centerline{Москва - 2024}
	\end{titlepage}
	\setcounter{page}{2}
	\tableofcontents
	\newpage
	
	\section{Постановка задач}
	\subsection{Задача 1}
	Исследовать электродинамику с монополями. Рассмотреть движение диона в однородном электрическом поле $E$; в однородном магнитном поле $B$; в скрещивающихся однородных электрическом и магнитном полях $E$ и $B$, причем считать, что $E \perp B$.
	\subsection{Задача 2}
	Исследовать модель Изинга для ферромагнетиков. Рассчитать вектор намагниченности, получить петлю гистерезиса (если возможно). 
	\subsection{Задача 3}
	Изучить движение заряженной частицы в равновесной электронейтральной плазме. Все необходимые параметры плазмы и частицы даны. 
	\newpage
	
	\section{Задача 1}
	\subsection{Электродинамика с монополями}
	Как известно, классические уравнения Максвелла несимметричны относительно обмена электрических и магнитных полей. Это, во многомо, связано с отсутствием магнитных зарядов. Однако существуют различные теории, которые предполагают их существование и позволяют исследовать так называемую электродинамику с монополями, чему и будет посвящен этот раздел. 

	Сперва следует договориться об обозначениях. Работать будем в системе СГС. К обозначениям будем добавлять индексы $\vec E$ или $\vec B$, в зависимости от того, с чем связан соответствующий объект (конкретная связь обычно будет понятна из контекста). 

	Начнем с получения новых уравнений Максвелла, а также преобразований обмена полей. Выпишем, для начала, классические уравнения Максвелла, с учетом соглашений об обозначениях: 
$$
\begin{aligned} 
\nabla \cdot \vec E & = 4\pi\rho_{\vec E} \\
\nabla \cdot \vec B & = 0 \\
\nabla \times \vec E & = -\frac{1}{c}\frac{\partial \vec B}{\partial t} \\
\nabla \times \vec B & = \frac{4\pi}{c}\vec j_{\vec E}+\frac{1}{c}\frac{\partial \vec E}{\partial t} \\
\end{aligned}
$$
Очевидно, что для получения симметричных уравнений, во второе из них необходимо добавить член, связанный с плотностью магнитных зарядов $\rho_{\vec B}$, а в третье \textbf{---} с током магнитных зарядов $\vec j_{\vec B}$. После их добавления получается следующая система уравнений:
$$
\begin{aligned} 
\nabla \cdot \vec E & = 4\pi\rho_{\vec E} \\
\nabla \cdot \vec B & =4\pi\rho_{\vec B} \\
\nabla \times \vec E & = -\frac{4\pi}{c}\vec j_{\vec B}-\frac{1}{c}\frac{\partial \vec B}{\partial t} \\
\nabla \times \vec B & = \frac{4\pi}{c}\vec j_{\vec E}+\frac{1}{c}\frac{\partial \vec E}{\partial t} \\
\end{aligned}
$$
При этом минус в третьем уравнении необходим из-за вида искомой симметрии. Полученная система уравнений оказывается симметрична относительно следующего преобразования:
$$
\begin{aligned} 
\vec E & \rightarrow \vec B \\
\vec B & \rightarrow -\vec E \\
\rho_{\vec E} & \rightarrow \rho_{\vec B} \\
\rho_{\vec B} & \rightarrow -\rho_{\vec E} \\
\vec j_{\vec E} & \rightarrow \vec j_{\vec B} \\
\vec j_{\vec B} & \rightarrow -\vec j_{\vec E} \\
\end{aligned}
$$
Опираясь на эти преобразования и на известные формулы классической электродинамики, можно получить следующие выражения:
$$\frac{\partial \rho_{\vec E}}{\partial t}+\nabla\cdot\vec j_{\vec E}=0 \rightarrow \frac{\partial \rho_{\vec B}}{\partial t}+\nabla\cdot\vec j_{\vec B}=0$$
$$\varphi_{\vec E} = \frac{q_{\vec E}}{r} \rightarrow \varphi_{\vec B} = \frac{q_{\vec B}}{r} $$
$$\vec E = \frac{q_{\vec E}}{r^3}\vec r \rightarrow \vec B = \frac{q_{\vec B}}{r^3}\vec r$$
$$\vec B=\frac{q_{\vec E}}{c}\frac{\vec v \times(\vec r - \vec{ r^\prime})}{|\vec r - \vec {r^\prime}|^3} \rightarrow \vec E=-\frac{q_{\vec B}}{c}\frac{\vec v \times(\vec r - \vec{ r^\prime})}{|\vec r - \vec {r^\prime}|^3}$$
$$\vec F_{\vec E}=q_{\vec E}(\vec E + \frac{1}{c}\vec v \times \vec B) \rightarrow \vec F_{\vec B}=q_{\vec B}(\vec B - \frac{1}{c}\vec v \times \vec E)$$
	\subsection{Движение диона в различных полях}
	\newpage
	
	\section{Задача 2}
	
	\newpage
	
	\section{Задача 3}
	
	
\end{document}